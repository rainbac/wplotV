\documentclass[12pt, parskip=full]{scrartcl}
\usepackage[utf8]{inputenc}
\usepackage[T1]{fontenc}
\usepackage{lmodern}
\usepackage{graphicx}
%\usepackage[ngerman]{babel}

%Math
\usepackage{amsmath}

\title{wplotView Userguide}
\author{Rainer Bachleitner}

\begin{document}
\maketitle

wplotView is a Python script to visualize Wannier functions generated by wien2wannier.  
It shows isosurfaces of the Wannier orbitals taking into account the phase of the complex wavefunction,
and the crystal structure.

It uses the ``Visualization Toolkit'' library within Python providing an easy, fast and open alternative to more fully featured
visualization applications.

\section{Basic Usage}

In addition to the *.struct file,wplotView needs the *.psink, *.psiarg and *.wplotout file produced by wplot. 
The program assumes that these files all share a common name, in the
following referred to as ``case''.

wplotView can be executed at the commandline as follows:
\begin{verbatim}
wplotView <case> [number of wannier function] [-up/-down] [-i isovalue] 
[-l/--nolabels] [--nooutline] [-nw/--nowindow] 
[--camera camera phase] [-c/--celllimits amount of cells along each axis]
[--multiply multiplication factor of screenshots]
\end{verbatim}

\paragraph{\texttt{<case>, number of wannier function, -up, -down}} 

These options tell the application for which files to look. For example
\begin{verbatim}
wplotView k555 7 -up 
\end{verbatim}
would look for k555.struct, k555.wplotoutup, k555\_7.psinkup and k555\_7.psiargup.
\begin{verbatim}
wplotView k555 1
\end{verbatim}
would look for k555.struct, k555.wplotout, k555\_1.psink and k555\_1.psiarg.

Omitting the number of the wannier function makes wplotView look for case.psink
and case.psiarg instead of case\_number.psink and case\_number.psiarg.

\paragraph{\texttt{-i isovalue}}

With this option the default isovalue for which the isosurface is plotted can
be overridden.  This is useful when changing the isovalue by pressing one of
the hotkeys (\emph{i} or \emph{o}, see section \ref{sec:hotkeys}) does not lead to the desired
isosurface fast enough and for quickly previewing isosurfaces for different
isovalues.

For now, providing a negative isovalue provides a workaround for displaying the crystal
structure without an isosurface.


\paragraph{\texttt{-l, -{}-nolabels}}
This disables labeling the atoms. Labels can also be turned on or off by
pressing \emph{l} while the program is running.

\paragraph{\texttt{-{}-nooutline}}
With this option set, the program doesn't show the bounding box of the
isosurface.

\paragraph{\texttt{-ptf, -{}-printtofile}}
This runs the program, sets the camera, takes a screenshot and stops the
program. This option is very useful in combination with -{}-camera and -c/-{}-celllimits 
because it enables the user to batch process many output files without requiring frequent
user interaction.

\paragraph{\texttt{-c/-{}-celllimits celllimits}}
By default, wplotView tries to calculate the amount of unit cells that have to
be displayed from the information given in the section ``Plotting area'' of the
*.wplotout file.

This option requires 6 integer numbers and tells wplotView how many unitcells
it is supposed to show. The first three numbers have to be zero or negative and
determine how many cells are created along the negative direction of the
coordinate axis, the second three numbers have to be zero or positive and
determine how many cells are created along the positive coordinate axes.

For example
\begin{verbatim}
--celllimits 0 0 0 1 1 1
\end{verbatim}
would tell wplotView that there should be one unit cell in the first octant of
the coordinate system in total.
\begin{verbatim}
--celllimits -1 -1 -1 2 2 2
\end{verbatim}
would create a 3 by 3 by 3 cuboid of unit cells.


%\paragraph{\texttt{-hq/-{}-highquality}}
%Experimental feature that enables some quality-improving features. 

\paragraph{\texttt{-ax/-{}-axes}}
Displays a coordinate system.

\paragraph{\texttt{-{}-camera camera phase}}

This option requires a set of cartesian coordinates and to additional numbers 
representing the position, the roll angle and the zoom factor of the
camera with respect to the isosurface center, which lies at the origin of the
coordinate system.

During the execution of wplotView, the camera phase is displayed in cartesian
coordinates along with the roll angle and the zoom factor
in the terminal output of the window and can be noted for later
reference.

Certain camera positions are hardcoded into wplotView and can be set by
pressing a hotkey (see section \ref{sec:hotkeys}).


\paragraph{\texttt{-{}-multiply multiplication factor}}

Providing a multiplication factor with this option overrides the default
multiplication factor of 2 when taking screenshots. This simply means that
screenshots will have the resolution of the wplotView window times the
multiplication factor.

Screenshots can be taken at runtime by pressing \emph{a}.

\section{Hotkeys} \label{sec:hotkeys}
The following keys have functions assigned to them when the program is running:
\paragraph{i and o} 
Pressing i or o changes the isovalue and updates the isosurface.
\paragraph{a}
Pressing a takes a screenshot and saves it to the folder the script is executed
in. If a multiplication factor is set (by default a multiplication factor of 2
is set), the screen acts up for a moment and then
goes back to normal. The screenshot will have the resolution of the wplotView
window times the multiplication factor.
\paragraph{l}
Pressing l turns the labels on or off.
\paragraph{1,2,4 and 5}
Pressing these keys puts the camera at certain hardcoded positions.
\paragraph{6,7}
Change the roll angle of the camera. Useful to adjust the scene.
\paragraph{h,j}
Changes the zoom level. \emph{h} increases the zoom level while \emph{j} 
reduces it.
\paragraph{q}
Pressing q quits wplotView.

\section{Mouse interaction}
Dragging the mouse across the screen while the left mouse button is pressed
will rotate the isosurface around the origin. Pressing the right mouse button
and moving the cursor up or down zooms in or out. Pressing the middle mouse
button and moving the mouse translates the whole scene across the screen.

Translating the scene across the screen will make it impossible to restore the
picture seen with the \verb+--camera+ option so it is not advised to do so.

\end{document}
